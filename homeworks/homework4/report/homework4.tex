\documentclass[letterpaper,12pt]{article}
\usepackage[T1]{fontenc}
\usepackage{mathptmx}

% unit definition
\def\degrees{\hbox{$^\circ$}}
\def\Celsius{\hbox{ $^\circ$C}}
\def\Rank{\hbox{ $^\circ$R}}
\def\Faren{\hbox{ $^\circ$F}}
\def\Kelvin{\hbox{ K}}
\def\inch{\hbox{ in}}
\def\ft{\hbox{ ft}}
\def\lbm{\hbox{ lb_{m}}}
\def\lbf{\hbox{ lb_{f}}}
\def\slug{\hbox{ slug}}
\def\psi{\hbox{ psi}}
\def\psf{\hbox{ psf}}
\def\pc{\hbox{ pc}}
\def\nm{\mbox{ nm}}

% hyperlink formatting
\usepackage{hyperref}
\hypersetup{
    colorlinks=true,
    linkcolor=red,
    urlcolor=purple,
    citecolor=blue
}

% Other general packages
\usepackage{pdfpages}
\usepackage{setspace}
\usepackage{graphicx}
\usepackage{float}
\usepackage{amsmath}
\usepackage{amssymb}
\usepackage{tabto}
\usepackage{booktabs, tabularx}
\usepackage{longtable}
\usepackage{enumitem}
\usepackage{gensymb}
\usepackage{cancel}
\usepackage{tikz}
\usepackage{rotating}
\usepackage{pgfplots}
\usepackage{appendix}
\usepackage[labelfont=bf, font={normalsize,stretch=1}]{caption}
\usepackage[letterpaper, margin=1.0in]{geometry}
\usepackage[utf8]{}
\usepackage{indentfirst}
\usepackage{lscape}
\usepackage{supertabular}
\usepackage{caption}
\usepackage{subcaption}

%Heading format
\usepackage{titlesec}
\titleformat*{\section}{\normalsize\bfseries}
\titleformat*{\subsection}{\normalsize\bfseries}
\titleformat*{\subsubsection}{\normalsize\bfseries\slshape}

%Page Numbers
\usepackage{fancyhdr} 
\pagestyle{fancy}
\fancyhf{}
\fancyheadoffset{0cm}
\renewcommand{\headrulewidth}{0pt} 
\renewcommand{\footrulewidth}{0pt}
\fancyhead[R]{\thepage}
\pagenumbering{arabic}

%listings package for code
\usepackage{listings}
\usepackage{xcolor}

% define table package
\usepackage{tabularx}
\newcommand\setrow[1]{\gdef\rowmac{#1}#1\ignorespaces}
% bibliography formatting
\usepackage{etoolbox}
\patchcmd{\thebibliography}{\section*{\refname}}{}{}{}

% color definitions
\definecolor{dblue}{HTML}{145680}
\definecolor{dred}{HTML}{801414}
\definecolor{dgreen}{HTML}{148014}
\definecolor{bgcode}{rgb}{0.95,0.95,0.95}
\definecolor{codegreen}{rgb}{0,0.6,0}
\definecolor{codegray}{rgb}{0.5,0.5,0.5}
\definecolor{codepurple}{rgb}{0.58,0,0.82}
\definecolor{backcolour}{rgb}{0.95,0.95,0.92}

\lstdefinestyle{mystyle}{
    backgroundcolor=\color{backcolour},
    commentstyle=\color{codegreen},
    keywordstyle=\color{magenta},
    numberstyle=\tiny\color{codegray},
    stringstyle=\color{codepurple},
    basicstyle=\ttfamily\footnotesize,
    breakatwhitespace=false,
    breaklines=true,
    captionpos=b,
    keepspaces=true,
    numbers=left,
    numbersep=5pt,
    showspaces=false,
    showstringspaces=false,
    showtabs=false,
    tabsize=2
}
\lstset{style=mystyle}

\pgfplotsset{compat=1.17}
\renewcommand{\thesubsection}{\thesection.\alph{subsection}}

% NOTES
%  - Double spacing (always fix for figures and tables though)
%  - for tables, remember to make them single spaced using \renewcommand{\arraystretch}{1}
%  - Always pull from GitHub to Overleaf when there are commits to be pulled (Menu > GitHub > PULL)
%  - for REFERENCES, use \cite{<label>} to link the reference.

% % TABLE TEMPLATE
% \begin{table}[H]
%     \begin{center}
%     \setstretch{1} 
%     \caption{\textbf{<caption here>}} \label{table:<label here>}
%     \begin{tabular}{|p{0.3in}|p{1in}|p{1in}|} % set column nums and width
%         \hline \textbf{No.} & \textbf{Item} & \textbf{Weight} \\ \hline % column headers
%         1 & Hot dogs & 2 lbs \\ \hline
%     \end{tabular}
%     \end{center}
% \end{table}

\begin{document}

%%% Title Pa
\begin{center}
    {\Large\textbf{AE 484 Homework 4}}\\
    Anshuk Chigullapalli, Max Kaiser, George Petrov, Kenneth Tochihara, Jeffery Zhou\\
\end{center}

% George, Anshuk, Kenneth
\section{Vehicle Design}

    % Anshuk
    \subsection{Fabrication Drawings}
    % Detail 3 view drawings (or two views when appropriate) view drawings of all fabricated components sufficiently dimensioned for manufacturing. Note that complex curves such as airfoils can show overall dimensions (like the chord length and thickness) with a leader and note to the surface indicating that the curve will be provided electronically. Be sure to name all parts and indicate the quantity and material they are made out of.
    
        The fabricated components of the aircraft include the  wings (and the cutouts in their foam), the vial release mechanism, the spars and cutouts for them, the elevons (and their actuator holders), the hand-holds, the motor retainer and and the winglets. Some of these components have complex geometries that will require the use of computer aided manufacturing such as 3D printing or CNC milling. The engineering drawings for all the relevant components are provided in the \ref{apx:eng_drawings}.
    
        Table \ref{table:fab_components} provides details on all fabricated components such as quantity and the material used.
        % insert images of the dimensioned drawings? Or create an appendix
        
        % add a table of all the fabricated parts
        
        \begin{table}[H]
            \begin{center}
            \caption{\textbf{Fabricated Components}} \label{table:fab_components}
             \begin{tabular}{|p{1.6in}|p{0.7in}|p{1.3in}|p{1.5in}|} % set column nums and width
                \hline \textbf{Component} & \textbf{Quantity} & \textbf{Material} & \textbf{Comments} \\ \hline % column headers
                Wing & 2 & Foam & Cutouts in the wing's foam are crucial and house all the flight critical components \\ \hline
                Elevon Actuator Holder & 2 & 3D Printed PLA & Connects the actuator rod to the elevon \\ \hline
                Trap Door & 1 & 3D Printed PLA & Part of Vial Release Mechanism \\ \hline
                Hand Holds & 2 & Foam + Composite & Allows for holding vehicle for hand launch \\ \hline
                Winglets & 2 & Foam + Composite & Attached to the primary wing using dowels \\  \hline
                Motor Retainer & 1 & 3D Printed PLA & Attaches the motor to the wing \\ \hline
            \end{tabular}
            \end{center}
        \end{table}
    
    % Anshuk
    \subsection{Full Assembly}
    % A fully assembled drawing showing the locations of all the parts and overall dimensions of key elements of the aircraft when it is put together. This should be presented both as a 3 view and isometric view drawing with all the parts named on the drawing.
    
        The team assembled all the individual components together in Siemens NX and created the full aircraft engineering drawing including the 3 view and isometric drawings, provided in \ref{apx:eng_drawings}. The overall dimensions can also be seen in the engineering drawings. There are three primary elements on the aircraft: the wings, the winglets and the elevons. The other components are electronics or actuators and are discussed in more detail in the following sections.
        
        Figure \ref{fig:iso_render} shows a CAD rendering of the complete aircraft in the isometric view. More renders are shown in Figures \ref{fig:bottom_render} and \ref{fig:closeup_render} to provide a better view of the aircraft.
          
        \begin{figure}[H]
            \centering
            \includegraphics[width=.9\textwidth]{homeworks/homework4/report/Figure/aircraft_assembly_front_iso.png}
            \caption{\textbf{Aircraft Isometric Render}}
            \label{fig:iso_render}
        \end{figure}
        
        \begin{figure}[H]
            \centering
            \includegraphics[width=1.1\textwidth]{homeworks/homework4/report/Figure/aircraft_assembly_2.png}
            \caption{\textbf{Aircraft Assembly Bottom View}}
            \label{fig:bottom_render}
        \end{figure}
        
        \begin{figure}[H]
            \centering
            \includegraphics[width=0.75\textwidth]{homeworks/homework4/report/Figure/aircraft_assembly_3.png}
            \caption{\textbf{Aircraft Assembly Close-up View}}
            \label{fig:closeup_render}
        \end{figure}
    
    % Kenneth
    \subsection{Equipment Locations}
    % A fully assembled drawing showing the locations of the equipment and instrumentation (components you buy and don’t fabricate like the receiver, servos, motor, battery, camera, video transmitter, ESC, etc.) on or within the airframe with key dimensions given for placement. This should be presented both as a 3 view and isometric view drawing and it may be useful to use cut-away views in some cases to better show where equipment/components are located.
    
        In Figure \ref{fig:component_locations}, locations of all commercial components are labeled in Table \ref{tab:component_locations}. \ref{apx:eng_drawings} shows the full engineering drawing of the render.
        
        \begin{figure}[H]
            \centering
            \includegraphics[width=0.9\textwidth]{homeworks/homework4/report/Figure/aircraft_assembly_underbelly_labeled.png}
            \caption{\textbf{Aircraft Commercial Component Locations}}
            \label{fig:component_locations}
        \end{figure}
        
        \begin{table}[H]
        \begin{center}
        \caption{\textbf{Component Locations}} \label{tab:component_locations}
        \begin{tabular}{|c|c|} 
            \hline
            \textbf{No.} & \textbf{Component} \\ \hline
            1 & Receiver \\ \hline
            2 & Vial Servo \\ \hline
            3 & Motor \\ \hline
            4 & Battery \\ \hline
            5 & ESC \\ \hline
            6 & +Y Elevon Servo \\ \hline
            7 & -Y Elevon Servo \\ \hline
            8 & 24" spar \\ \hline
            9 & 36" spar \\ \hline
        \end{tabular}
        \end{center}
        \end{table}
        
    % George
    \subsection{Exploded View}
    % An isometric exploded view assembly drawing with all the fabricated parts and purchased components labeled. No dimensions are needed for this drawing.
    
        In Figure \ref{fig:exploded}, an exploded view render of all the components can be seen. \ref{apx:eng_drawings}, shows the full engineering drawing of the render below.
        
        \begin{figure}[H]
            \centering
            \includegraphics[width=0.8\textwidth]{homeworks/homework4/report/Figure/aircraft_assembly_exploded.png}
            \caption{\textbf{Aircraft Exploded View}}
            \label{fig:exploded}
        \end{figure}
    
    % Theses drawings will be graded on how well and detailed you present your design. Include table of the characteristics of the tail and wing (airfoils, aerodynamic centers, S, MAC, sweep, taper, root chord, tip cord, twist, tail length etc.).
    
    % Kenneth
    \subsection{Design Overview}
    % Write a brief paragraph describing your design and the drawings presented.
    
        The aircraft was designed as a flying wing with tall winglets to add yaw stability to the flight. These winglets are secured with epoxied dowels to firmly secure the two components together. The main wing is assembled onto two carbon fiber spars. These lengths were determined based on the available at the source provided for carbon fiber tube purchases. Many of the components were mounted into the wing underneath to reduce protrusions throughout the body of the aircraft. Holes for servos, receiver, motor mount, battery, and vial mechanism are used to mount these components inside. Control surfaces and their actuators were included in the design to further increase the fidelity of model. A preliminary design of the hand holder was also included, with plans to refine the model in the future to incorporate throwing mechanics and ergonomics.
        
        Table \ref{table:design_char} provides details on all the other design characteristics of the aircraft.
        

% Anshuk    
\section{Vial Release Mechanism}
% Using figures and sentences/paragraphs describe your mechanism to release the protective vial described in the problem statement (cylinder 0.75” in diameter, 1.5” long and a mass of 5 grams) and how it works. What are the benefits and possible concerns you have with your design?

    The team decided to use a vial mechanism that was simple in design with minimal use of heavy parts or complex actuator mechanisms. The design was also kept within a size constraint so that the design would fit in front of the spars, helping keep the Center of Gravity in front of the Center of Pressure. 
    
    The design includes a 3D-printed plate that is connected to a servo arm placed next to the plate. The plate acts as a trap door at the bottom of the aircraft.The vial is placed on the flat plate in an upright position. When the servo actuates, the flat plate rotates with it away from the bottom and creates an opening out of the plane. The vial then falls out of the plane to be caught by someone on the ground.
    
    Figure \ref{fig:vial_assembly} shows the vial release mechanism assembly on its own and Figures \ref{fig:vial_closed} and \ref{fig:vial_open} shows the vial release assembly on the underside of the wing in the aircraft in both the open and closed configurations respectively.
    
    \begin{figure}[H]
        \centering
        \includegraphics[width=0.65\textwidth]{homeworks/homework4/report/Figure/assembly_vial_door.png}
        \caption{\textbf{Vial Release Mechanism}}
        \label{fig:vial_assembly}
    \end{figure}
            
    \begin{figure}[H]
        \centering
        \includegraphics[width=0.55\textwidth]{homeworks/homework4/report/Figure/aircraft_assembly_door_closed.png}
        \caption{\textbf{Vial Mechanism in the Closed Configuration}}
        \label{fig:vial_closed}
    \end{figure}
    
    \begin{figure}[H]
        \centering
        \includegraphics[width=0.55\textwidth]{homeworks/homework4/report/Figure/aircraft_assembly_door_open.png}
        \caption{\textbf{Vial Mechanism in the Open Configuration}}
        \label{fig:vial_open}
    \end{figure}
    
    This design involves only one moving part connected to the servo, and so was chosen for its simplicity and ease of building. The only harnessing required in this design would be for the servo, because the plate is attached to the servo itself and the vial sits freely. The primary concern is ensuring that the foam cut around the mechanism in the right way to account for the rotation of the flat plate. The team plans on fixing this concern by making the CAD accurate and fixing any tolerance issues so the fuselage foam can be cut accordingly.
    
% Jeff
\section{Free-Body Diagram}
% Using the weights and positions of all components (i.e. instrumentation, servos, motor, battery etc.) draw a free body diagram (using the CAD package) and calculate a force and moment balance using an appropriate reference point. Calculate the center of gravity of all the fabricated parts and components and list their coordinates relative to an appropriate reference point. Calculate the weight and center of gravity of the entire vehicle and write a short paragraph describing your analysis. 
    
    In this section, the team looked into the weights and positions of all components including the wing, spars, and the provided electronic components. By looking into the CAD drawings and using accurate measurement tools built into Siemens NX, the team was able to extract the precise locations of all important components along the chord-wise direction of the wing. Figure \ref{fig:free_body} represents a free body diagram of the aircraft and shows the major aerodynamic forces that the flying wing will experience through the flight and the location of which the forces are applied. The aircraft is simplified to a beam in this analysis to understand the overall loads across the $x$-axis.
    
    \begin{figure}[H]
        \centering
        \includegraphics[scale=0.6]{homeworks/homework4/report/Figure/fbd.png}
        \caption{\textbf{Free-body diagram of the flying wing}}
        \label{fig:free_body}
    \end{figure}
    
    As shown in the Figure \ref{fig:free_body}, the plane was designed so that the center of gravity is in front of the aerodynamic center, providing good aerodynamic stability for the flying wing. However, since the center of gravity ($x_{cg}$) is only marginally ahead of the aerodynamic center ($x_{ac}$), there may be some instability. Therefore, more analysis and reconfiguration will be needed in order to obtain an optimal stability for the aircraft. A more detailed diagram as shown below in Figure \ref{fig:weight_distribution} illustrates the chord-wise positions of all components on the aircraft so that the weight distribution can be better visualized. 
    
    Table \ref{tab:COG_calc} and Equation \ref{eqn:COG_calc} shows how the center of gravity ($x_{cg}$) of the aircraft was calculated. It was compared to the value computed using XFLR5 to validate the calculation in Section \ref{sec:flight_char}. 
    
    \begin{figure}[H]
        \centering
        \includegraphics[scale=0.52]{homeworks/homework4/report/Figure/components.png}
        \caption{This image displays the weight distribution of components on the wing.}
        \label{fig:weight_distribution}
    \end{figure}
    
    
    \begin{table}[H]
        \begin{center}
        \caption{\textbf{Component Masses and Locations}} \label{tab:COG_calc}
        \begin{tabular}{|c|c|c|}
        \hline
        \textbf{Component} & \textbf{Weight (g)} & \textbf{Distance from Front (mm)} \\ \hline
        Fudge Factor Mass & 32 & 0 \\ \hline
        Battery & 220 & 56.3  \\ \hline
        Vial Servo, Vial, ESC & 41 & 99.2 \\ \hline
        24 in Spar & 36.6  & 149.9 \\ \hline
        Receiver & 8 & 179.9 \\ \hline
        2 Elevon Servos & 72 & 267.5 \\ \hline
        36 in Spar & 39.4 & 269.9 \\ \hline
        Wing & 776 & 295 \\ \hline
        Motor, Propeller & 162 & 355 \\ \hline
        \end{tabular}
        \end{center}
    \end{table}
    
    Therefore, from the table shown above, the team is able to calculate the precise center of gravity, $x_{cg}$, location of the wing using Equation \ref{eqn:COG_calc}.
    
    \begin{equation} \label{eqn:COG_calc}
        x_{cg} = \frac{\sum_i m_i x_i }{\sum_i m_i} = 244.6 \text{ mm}
    \end{equation}
    
    As shown in the Table \ref{tab:COG_calc} above, a fudge factor mass was included to this aircraft to achieve a $x_{cg}$ that is in front of the $x_{ac}$ of the aircraft. Doing this would not exceed the overall mass of the aircraft as the 3 pound requirement will be met.
    
% Jeff
\section{Prominent Forces}

    Based on the force diagram shown in Figure \ref{fig:free_body}, the loads and moments displayed must be accounted for when designing a structural subsystem of the aircraft. The aircraft was designed in a manner where the aerodynamic loads are distributed through the two carbon fiber spars running across the span of the wing. One of these spars being located at the forward end of the aircraft, having a 0.5 in diameter and spanning 24 in, and the other spar being located at the aft end of the aircraft, having a 0.375 in diameter spars and spanning 36 in. The skin of the aircraft also plays a roll in enforcing the structure. 
    
    The major forces and moments which must be taken into account are the lift, thrust, and the resultant moment about the center of gravity. If the aircraft is split in half to analyze the lifting force on each side of the wing, the spars shall counteract in keeping the wind as a static member. The spars act as cantilever beams in this analysis. This can be see in Figure \ref{fig:wing_cant} where the red is the spar and the grey triangle is the wing. As demonstrated, the lift is being created and at the middle of the aircraft, but there is a counter moment to keep the structure static. The counter moment comes from the spar and is mirrored exactly on the $-y$ side of the aircraft.
    
    \begin{figure}[H]
        \centering
        \includegraphics[scale=0.7]{homeworks/homework4/report/Figure/forces_wing.png}
        \caption{\textbf{Force diagram of wing loading with spar}}
        \label{fig:wing_cant}
    \end{figure}
    
    The major areas of stress are located at the joint between each wing. This is where both sides of the wings are connected and in order to keep this region from failing, the spars act as continuous structural members within this connection. The elevon mechanism will also be under significant stress. Due to the high speeds of this aircraft, there is a risk that the actuating beam of the elevon mechanisms will buckle due to the counteracting aerodynamics forces on the elevon. 
    
    The battery, the vial, vial servo, and the ESC components are all positioned so that the their weight are loaded to the foam body. The elevon servo is loaded onto the foam. Finally, the motor and propeller are positioned so that the motor connector structure is directly screwed into the 36 in spar of which the thrusting force of the motor acts upon. Finally, the aerodynamic forces experienced by the elevons mostly act upon the foam and skin structures of the aircraft itself as mentioned above. Since the elevons' total mass is very small compared with the total mass of the aircraft, the team is confident that the plane structure is able to support those forces. 

% Based on your drawing write a short description indicating how the major forces and moments (i.e. weights, aerodynamic forces/moments, thrust etc.) will be structurally supported by your airframe design. Feel free to use neat simplified sketches and analysis to indicate how the stresses are carried by the airframe structure. Indicate the major regions of where stress concentrations may be the largest concern in your design.

% Max
\section{Design Characteristics}
% Provide tables of the characteristics [dimensions, airfoils, planform area, span, chord lengths, dihedral (if any), taper ratio, sweep, mean aerodynamic chord, root and mean chord length, elevon size (span and % chord), aerodynamic center and winglet or vertical stabilizer size (chord lengths, taper ratio etc.). write a short paragraph or two describing the process you used to determine the size and shape of the wing, elevons, and winglets. Justify why the wing elevons and winglets are the size and shape you have described. Also justify why you selected the airfoils that you are using.

    \begin{table}[H]
    \begin{center}
    \caption{\textbf{Design Characteristics of Aircraft}} \label{table:design_char}
    \begin{tabular}{|p{2.5in}|p{1.4in}|} 
        \hline
        \textbf{Characteristic} & \textbf{Value} \\ \hline
        Wingspan (in) & 69.5 \\ \hline
        Length (in) & 22.07 \\ \hline
        Airfoil & NACA 23012 \\ \hline
        Planform Area (in$^2$) & 748 \\ \hline
        Root Chord Length (in) & 14.5 \\ \hline
        Tip Chord Length (in) & 7.5 \\ \hline
        Taper Ratio & 6.182 \\ \hline
        Sweep (deg) & 22.019 \\ \hline
        Mean Aerodynamic Chord (in) &  11.371 \\ \hline
        Aerodynamic Center (in) & 9.635 \\ \hline
        Elevon Span (in) & 17 \\ \hline
        Elevon Chord Percentage (\%) & 18 \\ \hline
        Winglet Area (in$^2$) & 30 \\ \hline
        Winglet Airfoil & NACA 0010 \\
        \hline
    \end{tabular}
    \end{center}
    \end{table}

    The size and shape of the wing was chosen based on a few factors. The first choice made was in the wing span as the widest wings were desired to produce the most effective lift for the design. Then, the team decided that at least a 20 deg sweep was desired in order to ensure that there was adequate yaw stability without having to include massive winglets in the design. Having these two quantities specified, the airfoil was chosen in concert with choosing the root chord length, as the entire allotted thickness was desired in order to fit a majority of the components within the wing and without the need for an additional fuselage. The NACA 23012 airfoil was selected since it had this chamber thickness to fit these components and also created minimal moment about the center of gravity. The mean chord length was chosen based on the thickness of the NACA 23012 which offers a 1.75 in thickness at the root. From XFLR5 simulations, it was found that the NACA 23012 also has a center of pressure that is far back enough to be behind the $x_{cg}$, proving useful with the pusher motor configuration. The tip chord length was set at 7.5 to give sufficient room to attach adequately sized winglets. 

    After having chosen the wing design, the winglet design was the next area of focus. Since the sweep was already determined to be over 20\degree, large winglets were not necessary. An area of 30 in$^2$ was selected to generate some more lateral stability without adding too much weight to the ends of the wings. The airfoil chosen for the winglets was the NACA 0010 as this airfoil does not have any chamber and would more effectively provide yaw stability. As for the elevons, 50\% of the wingspan was selected to be utilized for the actuators and were biased towards the outer ends of the wings. The lecture slides were utilized to choose the elevon to chord ratio which stated that the chord shall be a range of 16\% to 20\%. 18\% was then chosen for this ratio.


% Max
\section{Flight Characteristics} \label{sec:flight_char}
% Write a paragraph and present calculations/analysis Indicating why your CG is located appropriately to achieve stability. Hint: Evaluate the longitudinal static margin of your simplified aircraft (i.e. wing without the winglets or vertical stabilizer) using XFLR5 and locating the neutral point. Also give conceptual arguments of the aircrafts lateral and directional stability. In addition to where you have placed the battery show that it can be adjusted along the axis (i.e. if possible it is good to allow the battery to be placed over the quarter chord (MAC) location on the wing). Provide a table of the overall geometric and flight characteristics of your aircraft (neutral point, SM, design α, design velocity, overall weight, etc.) and your best estimate velocity and angle of attack when hand launched. From your analysis provide graphs of (CD, CL, CL/CD, CM, vs. α) and describe their characteristics and what you conclude from them. Be sure to include paragraphs describing these results as they are presented and justifying your decisions.

    The team completed the XFLR5 model of the UAV both with and without winglets, with each model including the weights of all components in their respective placement. The final center of gravity location for the UAV ended up being 9.8 inches from the front of the UAV as determined by the XFLR5 center of gravity estimation. This was found to be sufficiently far forward as the center of pressure of the design stays behind this point up until about 10 deg AOA. This was as far forward as the center of gravity could be placed while still fitting all the necessary components within the wing itself. 
    
    The design will inherently be very stable in the laterally and longitudinally due to the sweep, winglets, and mass symmetry about the central axis. The sweep of the wings and the winglets will combine to provide ample yaw stability, while the mass symmetry will eliminate any moments about that center axis and provide the inherent roll stability. 
    
    The battery is placed as far forward as possible given the thickness of the front of the airfoil. If for any reason it needed to be moved farther back within the design, there is enough space for it to be swapped with the vial release mechanism. This would move the ESC and servo (with a combined weight of ~75 g) up to 2.25 in from the front and move the 180 g battery back to 3.5 in from the front. A picture of this area of the design is shown below, displaying that there is space for the battery to be moved if it were necessary.

    \begin{figure}[H]
        \centering
        \includegraphics[scale=0.5]{homeworks/homework4/report/Figure/battery_location.png}
        \caption{\textbf{Battery Maneuverability}}
        \label{fig:Battery Location}
    \end{figure}

    The majority of the flight characteristics were determined from analysis of the design as modeled in XFLR5. The design that was analyzed did not include the winglets as these caused errors due to interactions with the lift of the wing. The neutral point and center of gravity were given in the XFLR5 analysis. The design $\alpha$ was chosen as this angle gave a large increase in the lift of the wing over smaller angles, yet the drag is not as large as the higher angles of attack. The design velocity was chosen from the requirements for the plane design, as it was stated that the minimum top speed should be 40 mph. All of the corresponding flight characteristics in the table were found using a flight analysis of the design at the state design $\alpha$ and velocity of 1 deg and 40 mph respectively. The launch velocity is a conservative estimate assuming that the launcher will be able to throw the UAV with one hand similar to a football. The launch angle of 7.5 deg will provide sufficient lift at this low speed and can easily be adjusted by the thrower raising or lowering their aim.

    \begin{table}[H]
        \begin{center}
        \setstretch{1} 
        \caption{\textbf{Characteristics of Design}} \label{table:Weights}
        \begin{tabular}{|p{1.6in}|p{.8in}|} % set column nums and width 
        \hline
            \textbf{Characteristic} & \textbf{Value} \\ \hline
            Neutral Point (in) & 9.638 \\ \hline
            Center of Gravity (in) & 9.630 \\ \hline
            Design $\alpha$ (deg) & 1 \\ \hline
            CL at Design $\alpha$ & 0.169 \\ \hline
            CD at Design $\alpha$ & 0.010 \\ \hline
            CL/CD at Design $\alpha$ & 17.616 \\ \hline
            COP at Design $\alpha$ (in) & 10.189 \\ \hline
            Design Velocity (mph) & 40 \\ \hline
            Overall Weight (g) & 1357 \\ \hline
            Takeoff Velocity (mph) & 15 \\ \hline
            Takeoff $\alpha$ (deg) & 7.5 \\ \hline
        \end{tabular}
        \end{center}
    \end{table}

    \begin{figure}[h]
        \centering
        \begin{subfigure}{.5\textwidth}
          \centering
          \includegraphics[width=1\linewidth]{homeworks/homework4/analysis/CL vs AoA.png}
          \label{fig:CL vs. AoA}
        \end{subfigure}%
        \begin{subfigure}{.5\textwidth}
          \centering
          \includegraphics[width=1\linewidth]{homeworks/homework4/analysis/CD vs AoA.png}
          \label{fig:CD vs. AoA}
        \end{subfigure}
        \caption{\textbf{CL and CD vs. Angle of Attack}}
    \end{figure}
    
    \begin{figure}[h]
        \centering
        \begin{subfigure}{.5\textwidth}
          \centering
          \includegraphics[width=1\linewidth]{homeworks/homework4/analysis/CLCD vs AoA.png}
          \label{fig:CL/CD vs. AoA}
        \end{subfigure}%
        \begin{subfigure}{.5\textwidth}
          \centering
          \includegraphics[width=1\linewidth]{homeworks/homework4/analysis/CM vs AoA.png}
          \label{fig: CM vs. AoA}
        \end{subfigure}
        \caption{\textbf{CL/CD and CM vs. Angle of Attack}}
    \end{figure}

    Based on the graphs above, tested at the design velocity of 40 mph, the UAV design will be able to perform adequately throughout its flight profile. As can be seen in the CL/CD vs. AoA graph, it can be seen that this design produces at least 10x more lift than drag for the entirety of the positive range for the angle of attack. This will ensure that the design will produce enough lift to stay airborne in just about any situation. The other graphs simply show that the UAV will not exhibit any unusual stall characteristics or have any massive spikes in these flight characteristics throughout the anticipated flight envelope.

% Max
\section{Mass Estimates}
% Calculate the total weight of your aircraft (including the battery, motor, and vial delivery system) and describe how you determined it. What is the power to weight ratio and wing loading of your aircraft? Write a paragraph describing your analysis.

    \begin{table}[H]
        \begin{center}
        \setstretch{1} 
        \caption{\textbf{Weight of Aircraft Components}} \label{table:Weights}
        \begin{tabular}{|p{1in}|p{1in}|} % set column nums and width 
        \hline
            \textbf{Component} & \textbf{Weight (g)} \\ \hline
            Main Wings & 710 \\ \hline
            Winglets & 66 \\ \hline
            All equipment & 581 \\ \hline
            \textbf{Total Weight} & \textbf{1357} \\ \hline
        \end{tabular}
        \end{center}
    \end{table}

    The weight was calculated using the features XFLR5 provides. The mass of the wings and winglets were calculated using the wing area feature and the weight per unit area measurements provided in the lecture notes. Once these figures were collected, the point masses could be shifted within the software to represent the locations in reality. These were added in XFLR5 with placement as according to the CAD model, with the weights being figures taken either directly from course material or from the website with the listings of the carbon fiber tubes that are to be used for the spars. 45 grams of additional weight was added to account for any extra material that would connect our components back into the main spars. Given the wings, winglets, all equipment, and extra weight the final weight is 0.02 pounds under the 3 pound limit. 
    
    The wing loading of the design is about 2 g/in$^2$, which was determined using the wing loading feature given in XFLR5. The wing loading was also calculated by hand in \ref{apx:sample_calc}. The power to weight ratio is a little harder to determine. A power vs. voltage and amperage table was utilized from the motor provider. For this determination, a battery outputting 11.1 V and 24 A was assumed. With these values the motor will be outputting 266.4 W. Using this power and the aircraft weight of 1357 grams, a power to weight ratio of 0.196 W/g was determined. These calculations are shown in \ref{apx:sample_calc}.

\section*{Appendix A: Sample Calcs}

\begin{enumerate}[wide,label=\textbf{\arabic*}., labelindent=0pt]

    \item \textbf{Wing Loading}
        \[WS = \frac{W}{S_ref}\]
        $ W =$ weight [oz]\\
        $S_ref =$ strain in the $x$ direction after the transformation\\
        $\epsilon_y' =$ strain in the $x$ direction after the transformation\\
        
        \underline{For an aluminum beam in bending with a force of 8.90 N at Rosette 1:} \\
        \begin{align*}
            \gamma_{xy} &= (4.0 \times 10^{-6}) - (4.0 \times 10^{-6})\\
            &= 0.0 \times 10^{-6}\\
        \end{align*}
 
\section*{Appendix B: Engineering Drawings}\label{ApxB}   
\includepdf[pages=-,angle=90]{homeworks/homework4/report/Figure/gpetrov2_aircraft_assembly_exploded_drawing.pdf}
\end{enumerate}

\end{document}