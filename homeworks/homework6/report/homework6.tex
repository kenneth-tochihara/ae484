\documentclass[letterpaper,12pt]{article}
\usepackage[T1]{fontenc}
\usepackage{mathptmx}

% unit definition
\usepackage{siunitx}
% \sisetup{load-configurations = abbreviations}
\DeclareSIUnit\inch{in}
\DeclareSIUnit\ft{ft}
\DeclareSIUnit\Rank{^{\circ} R}
\DeclareSIUnit\Faren{^{\circ} F}
\DeclareSIUnit\lbm{lb_{m}}
\DeclareSIUnit\lbf{lb_{f}}
\DeclareSIUnit\torr{Torr}
\DeclareSIUnit\gallon{gal}
\DeclareSIUnit\slug{slug}
\DeclareSIUnit\knots{kts}
\DeclareSIUnit\miles{mi}


% hyperlink formatting
\usepackage{hyperref}
\hypersetup{
    colorlinks=true,
    linkcolor=red,
    urlcolor=purple,
    citecolor=blue
}


% Other general packages
\usepackage{setspace}
\usepackage{graphicx}
\usepackage{float}
\usepackage{amsmath}
\usepackage{amssymb}
\usepackage{tabto}
\usepackage{booktabs, tabularx}
\usepackage{enumitem}
\usepackage{gensymb}
\usepackage{cancel}
\usepackage{tikz}
\usepackage{pgfplots}
\usepackage{appendix}
\usepackage[labelfont=bf, font={normalsize,stretch=1}]{caption}
\usepackage[letterpaper, margin=1.0in]{geometry}
\usepackage[utf8]{}
\usepackage{indentfirst}
\setlength{\parindent}{0.25in}

%Heading format
\usepackage{titlesec}
\titleformat*{\section}{\normalsize\bfseries}
\titleformat*{\subsection}{\normalsize\bfseries}
\titleformat*{\subsubsection}{\normalsize\bfseries}

%Page Numbers
\usepackage{fancyhdr} 
\pagestyle{fancy}
\fancyhf{}
\fancyheadoffset{0cm}
\renewcommand{\headrulewidth}{0pt} 
\renewcommand{\footrulewidth}{0pt}
\fancyhead[R]{\thepage}
\pagenumbering{arabic}

%listings package for code
\usepackage{listings}
\usepackage{xcolor}

% bibliography formatting
\usepackage{etoolbox}
\patchcmd{\thebibliography}{\section*{\refname}}{}{}{}
\setstretch{2}

% color definitions
\definecolor{dblue}{HTML}{145680}
\definecolor{dred}{HTML}{801414}
\definecolor{dgreen}{HTML}{148014}
\definecolor{bgcode}{rgb}{0.95,0.95,0.95}
\definecolor{codegreen}{rgb}{0,0.6,0}
\definecolor{codegray}{rgb}{0.5,0.5,0.5}
\definecolor{codepurple}{rgb}{0.58,0,0.82}
\definecolor{backcolour}{rgb}{0.95,0.95,0.92}

\lstdefinestyle{mystyle}{
    backgroundcolor=\color{backcolour},
    commentstyle=\color{codegreen},
    keywordstyle=\color{magenta},
    numberstyle=\tiny\color{codegray},
    stringstyle=\color{codepurple},
    basicstyle=\ttfamily\footnotesize,
    breakatwhitespace=false,
    breaklines=true,
    captionpos=b,
    keepspaces=true,
    numbers=left,
    numbersep=5pt,
    showspaces=false,
    showstringspaces=false,
    showtabs=false,
    tabsize=2
}
\lstset{style=mystyle}

\pgfplotsset{compat=1.17}

% NOTES
%  - Double spacing (always fix for figures and tables though)
%  - for tables, remember to make them single spaced using \renewcommand{\arraystretch}{1}
%  - Always pull from GitHub to Overleaf when there are commits to be pulled (Menu > GitHub > PULL)
%  - for REFERENCES, use \cite{<label>} to link the reference.

% % TABLE TEMPLATE
% \begin{table}[H]
%     \begin{center}
%     \setstretch{1} 
%     \caption{\textbf{<caption here>}} \label{table:<label here>}
%     \begin{tabular}{|p{0.3in}|p{1in}|p{1in}|} % set column nums and width
%         \hline \textbf{No.} & \textbf{Item} & \textbf{Weight} \\ \hline % column headers
%         1 & Hot dogs & 2 lbs \\ \hline
%     \end{tabular}
%     \end{center}
% \end{table}

\begin{document}

%%% Title Pa
\begin{center}
    {\Large\textbf{AE 484 Homework 6}}\\
    Anshuk Chigullapalli, Max Kaiser, George Petrov, Kenneth Tochihara, Jeffery Zhou\\
\end{center}

\section{Changes Made}

    The team considered the changes suggested by Professor Gregory Elliott and modified the design of the aircraft accordingly. The primary changes made to the design are listed below. New engineering drawings are provided in the       of this document.
    
    \begin{enumerate}
        \item The winglets in the previous design had a curved merge into the wing. However, this is difficult in terms of manufacturability. The blend is replaced by a hard corner between the wing and the winglet.
        \item For ease of manufacturing, the elevons are extended to the end of the wing rather than being in the middle
        \item A plywood plate is added between the two wings to provide support for all the components and the motor mount.
        \item The spars are reduced in length because the original lengths were too long
        \item Changes to the design of the motor mount
    \end{enumerate}
    
    Another suggestion that was made was not using a 3D printed motor mount due to potential shear between the layers of the 3D print making it a bad material choice. The alternative would be a metal (potentially Aluminum) motor mount. However, the team decided to continue to go with the 3D printed mount for weight benefits and faster prototyping. If through physical test the team comes to the conclusion that the 3D printed motor mount would not work, the switch to a metal motor mount would be made.
    
    The drawings for the final aircraft are appended at the end of this document.

\section{Bill of Materials}

    \begin{table}[H]
    \centering
    \caption{Bill of Materials} \label{tab:bom}
    \begin{tabular}{|p{1.25in}|p{1.25in}|p{2.75in}|p{.75in}|}
    \hline
    \textbf{Item} & \textbf{Vendor} & \textbf{Model and Part \#} & \textbf{Price (\$)} \\
    \hline
    Front Spar & Goodwinds Composites & 0.5" x 24" Pultruded Carbon Tube,PCT500024 & 17.90   \\ \hline
    Back Spar     & Goodwinds& 0.375" x 24" Pultruded Carbon Tube, PCT375024 & 10.23   \\ \hline
    ESC           & Value Hobby & GForce 40A Brushless ESC, VHB-SC-4494 & 10.00 \\ \hline
    Battery       & Value Hobby & GForce 30C 2200mAh 3S 11.1V LiPo, RFI-LP-1899 & 14.90 \\ \hline
    Motor         & Value Hobby & GForce E480 Brushless Outrunner Motor, LEO-MT-3461 & 17.00   \\ \hline
    Propeller     & Horizon Hobby   & 12x6 3-Blade Pusher, MAS3B12X60R01 & 14.65 \\ \hline
    Receiver      & Horizon Hobby   & AR620 DSMX 6-Channel Sport Receiver, SPMAR620 & 49.99 \\ \hline
    Servos        & Value Hobby & Emax ES08MA II Analog Metal Gear Mirco Servo, VHB-SV-6148, Set of 4 & 17.50 \\ \hline
    Wing Cores & MOHR Composite & N/A & 90.00 \\ \hline
    Balsa Sheet & National Balsa & N/A & 18.00 \\ \hline
    Winglets & DragonPlate & 0 Degree Carbon Fiber Uni Sheet 1/8'' x 6'' x 6'', 2 sets & 55.62 \\ \hline
    Hand Holder & Amazon & Plywood Sheet 12'' x 12'' x 1/8 '' & 12.90 \\ \hline
    Motor Mount & 3D Printed & N/A & 0.44 \\ \hline
    Elevon Actuator Holder & 3D Printed  & N/A & 0.04 \\ \hline
    Trap Door     & 3D Printed  & N/A & 0.10    \\ \hline
    \end{tabular}
    \end{table}
    
\section{Manufacturing Files}
    
    The manfuacturing files for the laser cutter and 3D printers have been generated. They can be found \href{https://github.com/ktt3/ae484/tree/main/homeworks/homework6/manufacturing}{here} in our GitHub. The GCode for the 3D printers have already been generated for Max's CR6 printer.
    
    The final drawings for the laser cut, 3D print, and CNC foam parts are appended at the end of this document.

    \begin{table}[]
        \centering
        \caption{Manufactured Parts}
        \begin{tabular}{|c|c|c|c|}
            \hline
            \textbf{Part} & \textbf{Method} & \textbf{File Name} & \textbf{Weight} \\ \hline
            Winglets & Laser Cut & \verb|flat_winglets.pdf| & N/A \\ \hline
            Hand Holder & Laser Cut & \verb|holder_mid.pdf| & N/A \\ \hline
            Motor Mount & 3D Printed & \verb|motor_mount.stl| & 26 g \\ \hline
            Elevon Mount & 3D Printed & \verb|Elevon Mount.stl| & 1 g \\ \hline
            Vial Trap Door & 3D Printed & \verb|Trap Door.stl| & 5 g \\ \hline
            Main Wing +Z & CNC Foam & \verb|main_wing_+z.stl| & N/A \\ \hline
            Main Wing -Z & CNC Foam & \verb|main_wing_-z.stl| & N/A \\ \hline
        \end{tabular}
    \end{table}

\section{Table of Contributions}

    \begin{table}[H]
        \centering
        \caption{Table of Contributions}
        \label{tab:contributions}
        \begin{tabular}{|c|c|}
            \hline
            \textbf{Team Member} & \textbf{Contributions}  \\ \hline
            Anshuk Chigullapalli & \\ \hline
            George Petrov & \\ \hline
            Jeffery Zhou & \\ \hline
            Kenneth Tochihara & \\ \hline
            Max Kaiser & \\ \hline
        \end{tabular}
    \end{table}
    
\section*{Appended Drawings}

% From Q1

% From Q3
    
\end{document}